\documentclass[12pt,a4paper,english,oneside]{article}

% Valitse 'input encoding':
%\usepackage[latin1]{inputenc} % merkistökoodaus, jos ISO-LATIN-1:tä.
\usepackage[utf8]{inputenc}   % merkistökoodaus, jos käytetään UTF8:a
% Valitse 'output/font encoding':
%\usepackage[T1]{fontenc}      % korjaa ääkkösten tavutusta, bittikarttana
\usepackage{ae,aecompl}       % ed. lis. vektorigrafiikkana bittikartan sijasta
% Kieli- ja tavutuspaketit:
\usepackage[english]{babel}
% Muita paketteja:
% \usepackage{amsmath}   % matematiikkaa
\usepackage{url}       % \url{...}
\usepackage{verbatim} % comment block
\usepackage{enumitem} % lists with numbering
\setlist[enumerate]{label*=\arabic*.} % ^ global

% Kappaleiden erottaminen ja sisennys
\parskip 1ex
\parindent 0pt
\evensidemargin 0mm
\oddsidemargin 0mm
\textwidth 159.2mm
\topmargin 0mm
\headheight 0mm
\headsep 0mm
\textheight 246.2mm

\pagestyle{plain}

% Rangaistaan tavutusta
\hyphenpenalty=10000   % rangaistaan tavutuksesta, 10000=ääretön
\tolerance=1000        % siedetään välejä riveillä
% titlesec-paketti auttaa, jos tämän mukana menee sekaisin

% ---------------------------------------------------------------------

\begin{document}

\title{SCI3027.kand research plan:\\[5mm]
Utilisation of viewing statistics in detecting NPVR closing credits}

\author{Saskia Kivistö\\
Aalto University\\
\url{saskia.kivisto@aalto.fi}}

\date{\today}

\maketitle

% ---------------------------------------------------------------------

% käytä tässä dokumentissa numeroviitejärjestelmää komennolla \cite{kahva}.


\textbf{Title:} Utilisation of viewing statistics in detecting NPVR closing credits

\textbf{Author:} Saskia Kivistö

\textbf{Advisor:} Wanchote Jiamjitrak

%Tutkimusaiheen lyhyt kuvaus. Esittele aiheesi tiivistetysti.
\section{Abstract}

Network personal video recorder (NPVR) is a service for recording broadcast TV programs for later viewing. Instead of storing recordings on the users local device, NPVR stores recordings on the content providers server. For every program listed in the Electronic program guide (EPG), a single recording is created and stored on the server. The users who record the same program will receive the same video from the server.

The recording start and end times are determined by the scheduling information given in the EPG. However, it is common that programs are not broadcasted exactly according to the EPG schedule. Thus NPVR recordings typically have some non-program content at the beginning and end. The goal of this thesis is to study whether viewing statistics can be used in determining where the actual program content ends and where the closing credits begin in a NPVR recording.

%Mitä haluat saada selville? Mitkä ovat keskeisiä kysymyksiä? Mistä näkökulmasta asiaa tarkastellaan?
%Tutkimuskysymystä kannattaa siis rajata ja tarkentaa sekä huomioida näkökulman merkitys. Ts. jänikset eläintieteen kannalta ovat eri aihe kuin jänikset metsästyksen näkökulmasta.
\section{Goals and perspectives}

%Millaisen aineiston varaan perustat tutkimuksesi? Arvioi materiaalin riittävyyttä asetettuihin tavoitteisiin nähden.
%Pitää olla siis hieman kuvaa siitä, minkälaisen materiaalin kanssa ollaan tekemisissä ja mitä sellaisen käsittelyyn tarvitaan (etenkin siis tarvittavan ajan puolesta; ts. kuinka monta tuntia/minuuttia per lähde?).
\section{Research material}

The literature review part can be divided into three kind of subjects:

\begin{enumerate}
    \item Previous research on methods for video content detection and labeling (closing credits, ads, etc.), to give more context on the thesis topic
    \item Research on approaches to find the location of a specific pattern from data (in this case the drop in viewer count during closing credits). Needed for choosing a good model for deciding where the end credits begin based on the viewer count. 
    \item Research on the best practices of model validation. Needed for choosing a reasonable method for validating the results given by the labeling model.
\end{enumerate}

%Miten sen keräät materiaalisi tai saat sen käsiisi? Kuinka käsittelet sen? Kuinka siitä tulee raportti?
%Tavallaisesti kirjallisuustutkimuksen yhteydessä tämä on:
%(a) lähderyhmien valinta,
%(b) viitteiden ja lähteiden haku,
%(c) lähteiden arviointi,
%(d) lähteiden lukeminen,
%(e) tiedon organisointi,
%(f) raportointi.  % (f) tärkeää ettei jää vain lukemiseksi!
%Kirjallisuustutkimuksen yleinen menetelmä pitää sovittaa tähän nimenomaiseen aiheeseen sekä tekijän lähtökohtiin. Kuinka sinä teet muistiinpanot (että myös kirjoitat etkä pelkästään lue). Eli tälle pitää hieman miettiä omakohtaista vaiheistusta. Siis nähdä ihan oikeasti, kuinka sinä saat tutkielman tehtyä.
%Ja... raportointi ei ole kirjoittamista vaan jo kirjoitettujen muistiinpanojen koostamista yhteneväksi teokseksi.
\section{Research methods}

%will follow the standard methods of empirical research
%refutable hypothesis etc. (look into this)

%Yleensä kaikkiin töihin liittyy kompastuskiviä. Ne on syytä tiedostaa etukäteen. Yhdessä työssä aihe on suurpiirteinen (työn rajaaminen vaikeaa), toisessa materiaalia on niukasti saatavissa, kolmannessa taas materiaalia on hukkumiseen asti.  Eli, nämä pitäisi kyseisen tutkimuksen osalta kirjata ylös, ja nähdä ne myös mahdollisuuksina (positiivisina haasteina) ei ainostaan esteinä.
\section{Challenges}

%data collection
The validation of the experimental work relies on collecting the end times of program recordings manually. This is somewhat time consuming. Part of the literature review could maybe be finding papers that discuss what is a sufficient amount of data for validation.

%data being deleted
Viewing statistics are stored for 100 days. This shouldn't be a problem if I follow the bachelor's thesis course scedule, but if writing the thesis takes longer this might become a problem. In order to have access to all viewings of a recording during the course of the entire writig process, I'm using only recordings that have been broadcasted in 2022.


%Kuka tätä työtä tekee, kuka ohjaa, jne. Paljonko on käyttää aikaa. Tarvitaanko muuta? (Onko työssä joku kokeellinen osuus?)
\section{Resources}

The thesis topic is from a company. I have a full time contract for this spring, and I have agreed with my line manager that I can focus mainly on this thesis during the spring. The thesis will include an applied part, and the company will provide me access to the data needed for it.

%Laadi tutkimustyölle ja raportoinnille realistinen aikataulu. Huolehdi, että suunnitelmasi vastaa kandidaatin tutkielman sekä seminaarin aikataulua sekä laajuutta.  \emph{Kurssiesitteessä omalle kirjoitusprosessille on arvioitu noin 6 op eli 160 tuntia eli noin 4 viikkoa työtä.}
\section{Schedule}

Course scedule:

\begin{tabular}{|p{20mm}|p{130mm}|}
\hline
8.2.   & 1st draft ready, 2-3 pages \\ \hline
1.3.   & 2nd draft ready, $\sim$10 pages \\ \hline
22.3.  & 3rd draft ready \\ \hline
11.4.-14.4. & presentation at the final seminar \\ \hline
17.4.  & submission of the final version \\ \hline
\end{tabular}
\\
\\
\\
\textbf{Manually collecting program end times for validation:}\\
From now on until the end of March. A menial, but time consuming task.
\\
\\
\textbf{Finding, reading and taking notes on relevant research}:\\
From now on until the middle of February.  
\\
\\
\textbf{Using the methods discussed in previous research for this use case:}\\
From the middle of February to the end of March. I'm aiming to have some results ready for the 2nd draft and to have almost complete results for the 3rd draft.
\\
\\
\textbf{Finishing the thesis:}\\
From 3rd draft until the submission of the final version.


%Laadi lyhyt sisällysluettelo, jossa on hahmoteltuna kandityön pää- ja alaluvut. Yleensä perusrunko on
%(1) Johdanto,
%(2) Tausta,
%(3) Sisäluvut,
%(4) Yhteenveto.
%Sinun täytyy suunnitella oma raportointisi tähän sopivaksi. 
%\emph{Rakenne tarkentuu työn edetessä. Tutkimussuunnitelmaan ei välttämättä tarvita lähdeluetteloa, mutta halutessasi voit sisällyttää tärkeimmät lähteet.}
\section{Structure of the thesis}

\begin{enumerate}
\item Introduction 
\begin{enumerate}[label=${}$]
    \item Terminology, the goal and the scope of the thesis
\end{enumerate}
\item Background (literature review) 
    \begin{enumerate}
    \item Previous research on methods for video content detection and labeling
    \\(closing credits, ads, etc.)
    \item Research on approaches to find the location of a specific pattern from data
    \\(in this case the drop in viewer count during closing credits)
    \item Research on the best practices of model validation
    \end{enumerate}
\item Research methods and data (own research)
\begin{enumerate}[label=${}$]
    \item Trying out one or several of the approaches discussed in section 2.2,
    \\validating the results according to section 2.3
\end{enumerate}
\item Results
\item Conclusions
\end{enumerate}

% ---------------------------------------------------------------------
%
% ÄLÄ MUUTA MITÄÄN TÄÄLTÄ LOPUSTA

% Tässä on käytetty siis numeroviittausjärjestelmää. 
% Toinen hyvin yleinen malli on nimi-vuosi-viittaus.

% \bibliographystyle{plainnat}
%\bibliographystyle{finplain}  % suomi
\bibliographystyle{plain}    % englanti
% Lisää mm. http://amath.colorado.edu/documentation/LaTeX/reference/faq/bibstyles.pdf

% Muutetaan otsikko "Kirjallisuutta" -> "Lähteet"
\renewcommand{\refname}{References}  % article-tyyppisen

% Määritä bib-tiedoston nimi tähän (eli lahteet.bib ilman bib)
%\bibliography{lahteet}

% ---------------------------------------------------------------------

\end{document}

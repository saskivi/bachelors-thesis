% Tiivistelmät tehdään viimeiseksi. 
%
% Tiivistelmä kirjoitetaan käytetyllä kielellä (JOKO suomi TAI ruotsi)
% ja HALUTESSASI myös samansisältöisenä englanniksi.
%
% Avainsanojen lista pitää merkitä main.tex-tiedoston kohtaan \KEYWORDS.

\begin{enabstract}

The goal of this thesis was to study whether it is possible to use user viewing behaviour statistics to distinguish program content from surplus content from television broadcasts, which have been recorded with a network personal video recorder. Recorded television programs tend to include surplus content, since programs are not broadcast strictly according to the electronic program guide. In order to ensure that the entire program is recorded, recording must start a few minutes earlier and end a few minutes later than what is stated as the programs broadcast time in the electronic program guide. The incentive for identifying surplus content is that it consumes storage space, and fast forwarding over it is tedious for the person watching the recording.

The research question was treated as a signal processing problem for offline change point detection, more specifically as a minimisation problem for the sum of costs of segments. %The homogeneity of possible segments was measured with a cost function of least squares deviation, and the optimal segmentation was computed with dynamic programming. % two different dynamic programming methods
The optimal segmentation was computed with dynamic programming, using least squares deviation as the cost function.

The position of opening and closing credits was correctly identified for most of the recordings in the sample of 279 recordings. The last change point was typically within a few seconds from the beginning of closing credits, but for recordings with program content during the closing credits, the last change point was predicted closer to the end of closing credits. There was more variance in the position of the first change point within opening credits.

The precise start and end time of program content could not be detected solely with user viewing behaviour statistics, but the data could be useful for more precise detection combined with other analytics and information about the typical program structure.

\end{enabstract}

\begin{fiabstract}

%TODO: aikamuodot
%TODO: englanninkieliset termit mukaan sulkuihin

% (1) aihe, tavoite ja rajaus (heti alkuun, selkeästi ja napakasti, ei johdattelua)
%kommentti: Tässä käydään asiaan hyvin nopeasti, eikä lukija välttämättä tajua, mistä tutkimuksessa on kyse. Asiaa voisi avata ehkä hieman rauhallisemmin.
Tässä kandidaatintyössä tutkittiin, onko katsomiskäyttäytymistilastojen avulla mahdollista erottaa ohjelmasisältö ylimääräisestä sisällöstä televisiolähetyksissä, jotka on nauhoitettu verkkopohjaisella digitaalisella mediatallentimella (engl. network personal video recorder). Nauhoitteiden alkuun ja loppuun jää ylimääräistä sisältöä, koska nauhoituksen tallennusaika on muutaman minuutin pidempi kuin ohjelman ohjelmaoppaan mukainen kesto. Tällä varmistetaan, että ohjelma tallentuu kokonaisuudessaan, vaikka sen lähetysaika poikkeaisikin hieman ohjelmaoppaasta.
Kannuste ylimääräisen sisällön tunnistamiselle on, että se vie tallennustilaa ja sen ohi kelaaminen on puuduttavaa.

% (2) aineisto ja menetelmät (erittäin lyhyesti)
Tutkimuskysymystä lähestyttiin signaalinkäsittelyn kautta signaalin segmentiointiongelmana. Mahdollisten segmenttien homogeenisyyttä mitattiin neliöllisellä tappiofunktiolla. 
Tappiofunktion minimoivan segmenttiyhdistelmän laskemiseen sovellettiin kahta algoritmia, joista ensimmäinen löytää optimaalisen ratkaisun, kun segmenttien määrä on määritelty ennalta, ja toinen, kun segmenttien määrä ei ole tiedossa.

% (3) tulokset (tälle enemmän painoarvoa)
%kommentti: Lisää tähän metatekstiä: Tutkimuksessa alku- ja lopputekstien...
Alku- ja lopputekstien sijainti löytyi oikein suurimmassa osassa 279 tallenteen otoksesta. Lopun muutoskohta paikantui tyypillisesti muutaman sekunnin päähän lopputekstien alkukohdasta, mutta jos lopputekstien aikana tai päätteeksi oli vielä varsinaista ohjelmasisältöä, muutoskohta paikantui lähemmäs lopputekstien loppua. Alun muutoskohdan sijainnissa alkutekstien sisällä oli enemmän vaihtelua.

% (4) johtopäätökset (tälle enemmän painoarvoa)
Alku- ja lopputekstien tarkkaa alku- ja päätöskohtaa ei pysty määrittämään pelkkien katsomistilastojen avulla. Katsomistilastot voisivat kuitenkin olla hyödyllisiä tarkan alun ja lopun paikantamisessa yhdistettynä muuhun analytiikkaan ja metatietohin ohjelmien tyypillisestä rakenteesta.

\end{fiabstract}

% Tiivistelmä on muusta työstä täysin irrallinen teksti, joka kirjoitetaan tiivistelmälomakkeelle vasta, kun koko työ on valmis. Se on suppea ja itsenäinen teksti, joka kuvaa olennaisen opinnäytteen sisällöstä. Tavoitteena selvittää työn merkitys lukijalle ja antaa yleiskuva työstä. Tiivistelmä markkinoi työtäsi potentiaalisille lukijoille, siksi tutkimusongelman ja tärkeimmät tulokset kannattaa kertoa selkeästi ja napakasti. Tiivistelmä kirjoitetaan hieman yleistajuisemmin kuin itse työ, koska teksti palvelee tiedonvälitystarkoituksessa laajaa yleisöä.

% Tiivistelmän rakenne: teksti jäsennetään kappaleisiin (3-5 kappaletta); ei väliotsikkoja; ei mitään työn ulkopuolelta; ei tekstiviitteitä tai lainauksia; vähän tai ei ollenkaan viittauksia työhön (ei ollenkaan: “luvussa 3” tms., mutta koko työhön voi viitata esim. sanalla “kandidaatintyössä”; ei kuvia ja taulukoita.

% Tiivistelmässä otetaan “löysät pois”: ei työn rakenteen esittelyä; ei itsestäänselvyyksiä; ei turhaa toistoa; älä jätä lukijaa nälkäiseksi, eli kerro asiasisältö, älä vihjaa, että työssä kerrotaan se.

% Tiivistelmän tyypillinen rakenne:
% (1) aihe, tavoite ja rajaus (heti alkuun, selkeästi ja napakasti, ei johdattelua)
% (2) aineisto ja menetelmät (erittäin lyhyesti)
% (3) tulokset (tälle enemmän painoarvoa)
% (4) johtopäätökset (tälle enemmän painoarvoa)
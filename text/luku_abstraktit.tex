% Tiivistelmät tehdään viimeiseksi. 
%
% Tiivistelmä kirjoitetaan käytetyllä kielellä (JOKO suomi TAI ruotsi)
% ja HALUTESSASI myös samansisältöisenä englanniksi.
%
% Avainsanojen lista pitää merkitä main.tex-tiedoston kohtaan \KEYWORDS.

\begin{enabstract}
Abstract
\end{enabstract}

\begin{fiabstract}

% (1) aihe, tavoite ja rajaus (heti alkuun, selkeästi ja napakasti, ei johdattelua)
Tässä kandidaatintyössä tutkitaan onko katsomiskäyttäytymistilastojen avulla mahdollista erottaa ohjelmasisältö ylimääräisestä sisällöstä televisiolähetyksissä, jotka on nauhoitettu verkkopohjaisella digitaalisella mediatallentimella. Nauhoitteiden alkuun ja loppuun jää ylimääräistä sisältöä, koska nauhoituksen tallennusaika on muutaman minuutin pidempi kuin ohjelman ohjelmaoppaan mukainen kesto. Tällä varmistetaan että ohjelma tallentuu kokonaisuudessaan, vaikka sen lähetysaika poikkeaisikin hieman ohjelmaoppaasta.
Kannuste ylimääräisen sisällön tunnistamiselle on, että se vie tallennustilaa ja sen ohi kelaaminen on puuduttavaa.

% (2) aineisto ja menetelmät (erittäin lyhyesti)
Tutkimuskysymystä lähestytään signaalinkäsittelyn kautta segmentiointiongelmana. Mahdollisten segmenttien homogeenisyyttä mitataan neliöllisellä tappiofunktiolla. 
Tappiofunktion minimoivan segmenttiyhdistelmän laskemiseen sovelletaan kahta algoritmia, joista ensimmäinen löytää optimaalisen ratkaisun kun segmenttien määrä on määritelty ennalta ja toinen kun segmenttien määrä ei ole tiedossa.

% (3) tulokset (tälle enemmän painoarvoa)
Alku- ja lopputekstien sijainti löytyy oikein suurimmassa osassa 279 tallenteen otoksesta. Lopun muutoskohta paikantuu tyypillisesti muutaman sekunnin päähän lopputekstien alkukohdasta, mutta jos lopputekstien aikana tai päätteeksi on vielä varsinaista ohjelmasisältöä muutoskohta paikantuu lähemmäs lopputekstien loppua. Alun muutoskohdan sijainnissa alkutekstien sisällä on enemmän vaihtelua.

% (4) johtopäätökset (tälle enemmän painoarvoa)
Alku- ja lopputekstien tarkkaa alku- ja päätöskohtaa ei pysty määrittämään pelkkien katsomistilastojen avulla. Katsomistilastot voisivat kuitenkin olla hyödyllisiä tarkan alun ja lopun paikantamisessa yhdistettynä muuhun analytiikkaan ja metatietohin ohjelmien tyypillisestä rakenteesta.

\end{fiabstract}

% Tiivistelmä on muusta työstä täysin irrallinen teksti, joka kirjoitetaan tiivistelmälomakkeelle vasta, kun koko työ on valmis. Se on suppea ja itsenäinen teksti, joka kuvaa olennaisen opinnäytteen sisällöstä. Tavoitteena selvittää työn merkitys lukijalle ja antaa yleiskuva työstä. Tiivistelmä markkinoi työtäsi potentiaalisille lukijoille, siksi tutkimusongelman ja tärkeimmät tulokset kannattaa kertoa selkeästi ja napakasti. Tiivistelmä kirjoitetaan hieman yleistajuisemmin kuin itse työ, koska teksti palvelee tiedonvälitystarkoituksessa laajaa yleisöä.

% Tiivistelmän rakenne: teksti jäsennetään kappaleisiin (3-5 kappaletta); ei väliotsikkoja; ei mitään työn ulkopuolelta; ei tekstiviitteitä tai lainauksia; vähän tai ei ollenkaan viittauksia työhön (ei ollenkaan: “luvussa 3” tms., mutta koko työhön voi viitata esim. sanalla “kandidaatintyössä”; ei kuvia ja taulukoita.

% Tiivistelmässä otetaan “löysät pois”: ei työn rakenteen esittelyä; ei itsestäänselvyyksiä; ei turhaa toistoa; älä jätä lukijaa nälkäiseksi, eli kerro asiasisältö, älä vihjaa, että työssä kerrotaan se.

% Tiivistelmän tyypillinen rakenne:
% (1) aihe, tavoite ja rajaus (heti alkuun, selkeästi ja napakasti, ei johdattelua)
% (2) aineisto ja menetelmät (erittäin lyhyesti)
% (3) tulokset (tälle enemmän painoarvoa)
% (4) johtopäätökset (tälle enemmän painoarvoa)
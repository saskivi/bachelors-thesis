\section{Introduction} \label{sec:intro}

% lineeaari tv -> videonauhuri -> on demand -> on demand tv
Before videocassette recorders became widely available for regular households, watching television was a very time-sensitive activity: if you wanted to see a TV program you had to watch it when it was broadcasted. When video recorders became affordable, they enabled people to record and watch TV programs whenever they wanted. Currently, linear TV is losing even more of its foothold, as various video-on-demand services are gaining popularity. % at the expense of broadcast TV.
Recording storage is also moving away from the homes of viewers into the cloud servers of content providers.

% NVPR = videonauhuri pilvessä, jonka katsoja jakaa muiden kanssa
Network personal video recorder (NPVR) is a type of service for recording broadcast TV programs for later viewing. Instead of storing recordings on the users local device, NPVR stores recordings on the content providers server. For every program listed in the Electronic program guide (EPG), a single recording is created and stored on the server. The users who record the same program will receive the same video from the server.

% NPVR ongelma
The NPVR recording start and end times are determined by the scheduling information given in the EPG. However, it is common that programs are not broadcasted exactly according to the EPG schedule. To ensure that the whole program is recorded, some margin is typically added on both ends of the recording. Thus, NPVR recordings tend to have some non-program content at the beginning and end. The goal of this thesis is to study whether user viewing behaviour can be used in determining where the closing credits begin in a NPVR recording.

% motivaatio
I am writing this thesis for a  NPVR service provider company. From the perspective of a NPVR service provider, closing credits detection has a few uses. Firstly, less storage space is needed if the surplus contet after the closing credits is discarded. Secondly, when the customers want to watch multiple episodes of a series in a row, it is convenient for them if a link to the next episode is displayed during the closing credits.

% rajaus
% User viewing behaviour might also be useful for detecting starting credits and advertisement breaks, but on this thesis I will focus on the closing credits detection to narrow down the topic. I will also restrict the examined recordings to TV series with multiple episodes and at least one hundred views per episode.

% sisältö ja rakenne
The main goal of this thesis is to study whether user viewing behaviour can be used for detecting closing credits of NPVR recordings. To support the empirical research, the chapter on theoretical background discusses signal change detection, and result evaluation and validation. The empirical research chapter is divided into four section. The ground truth is defined in the first section. The second section discusses the properties of the user viewing behaviour data. In the third section the Python scientific library \texttt{ruptures} is used to detect closing credits. The results are evaluated in the fourth section. The discussion chapter considers, based on the previous chapters, the viability of using user viewing behaviour for closing credits detection. Lastly, the main points of this thesis are summarised in the conclusions.

\section{Theoretical background} \label{sec:background}

\subsection{Signal change detection} \label{subsec:methods} % for this specific case

%definition
%classification:
%methods (the paper, bayes, something else)
%online/offline
%(un)known nof points 
%cost function, parametric/non-parametric
%search method, optimal/approximate (accuracy vs performance)

%tie to npvr case & definition
Locating the closing credits from user behaviour data can be formulated as a signal processing problem, more precisely change point detection problem. Signal change point detection is quite widely researched topic, since it has applications in multiple fileds such as ... % TODO: add examples

% offline
Change point detection problems can be divided into two main categories, depending on whether the change detection must be done for incoming data in near real-time, or not. Methods solving the former case are referred to as online algorithms. Offline algorithms solve the latter case, and they differ from the online algorithms by getting the entire dataset as input and typically being more computationally complex, but also by detecting the changes more accurately.

% unknown number of changes
Offline methods can be divided into two categories, based on whether the number of changes in the dataset is known beforehand, or not. If the number of change points is not known, an extra step is needed to determine it. There are multiple methods for doing this.


\subsection{Result evaluation and validation} \label{subsec:validation}

% specific to change point detection:

% Annotation error
% kuinka paljon mallin antamien muutospisteiden määrä eroaa todellisesta määrästä

% Hausdorff
% kuinka suuri on isoin ero mallin antamasta muutoskohdasta lähimpään todelliseen muutoskohtaan

% Rand index
% kuinka suuren osan koko otoksesta malli on luokitellut oikein

% F1-Score
% precision (kuinka suuri osa löydetyistä todellisia), recall (kuinka suuri osa todellisista löydetty)

% general:

\section{Empirical research} \label{sec:casestudy}

\subsection{Ground truth} \label{subsec:groundtruth}

In order to evaluate how well a method detects NVPR closing credits, a ground truth is needed for comparison. The ground truth is obtained by direct human visual observation, meaning that a person must look at a video recording and write down the timestamps where the closing credits begin and end.

I have collected the closing credits start and end time from 555 NPVR recordings by hand with a margin of error of $\pm$ 1 seconds. Each of these recordings has at least one hundred user views.

\begin{figure}[H]
    \centering
    \includegraphics[width=1\textwidth]{../plots/episode.png}
    \caption{Visualisation of user views count for each second in an example NVPR recording}
    \label{fig:intro_ads_outro}
    \end{figure}

\subsection{User viewing behaviour data} \label{subsec:data}
% TODO: rephrase sentences with "second-long" to be clearer
Plotted in Figure \ref{fig:intro_ads_outro} is a sample of 92 user views of a NPVR recording. The horisontal axis contains one second long intervals, corresponding to each second of the recording. The vertical axis shows how many times a second-long interval in the recording was viewed by users. For example, 58 users from the sample of 92 users watched the part of the video between 0:10:00 - 0:10:01 (600 on the horisontal axis in Figure \ref{fig:intro_ads_outro}).

The red colored part of the plot in Figure \ref{fig:intro_ads_outro} is where the starting credits, advertisement break and closing credits of the program are. I checked the location of the aforementioned by looking at the recorded video. A steep increase in views can be seen in the plot whenever the actual program content begins, and correspondingly there is a steep decrease in views when the program content shifts into to advertisements or closing credits.

\subsection{Closing credits detection with \texttt{ruptures} library} \label{subsec:solution}
% cost-function -> probably non-parametric (distribution not known)
% search method -> probably optimal (pelt) but approximate methods could be tried
% constraint -> unknown K^* 

A Python library called \texttt{ruptures} will be used for the change point detection. \texttt{ruptures} is based on the findings of a literature reviw conducted by Truong et al. \cite{truongSelectiveReviewOffline2020} which examines various methods for offline change point detection. Selected algorithms examined in the literature review are implemented in \texttt{ruptures}.

Choosing the most suitable algorithm %from the library
for this use-case can be done by considering the three aspects of change detection methods discussed in the literature review: cost-function, search method and constraint. The cost-function must be non-parametric, since the probability distributions of the data are unknown. Accuracy is more important than low computational complexity, so optimal method is preferable to an approximate one. As the number of change points is unknown, the only suitable optimal search method is \texttt{Pelt}. The algorithm was first indroduced by Killick et al. \cite{killickOptimalDetectionChangepoints2012}.

An example of \texttt{Pelt} output is visualised in Figure \ref{fig:ruptures_change_detection}. The input data is the same as in Figure \ref{fig:intro_ads_outro}. Red rectangles on the plot represent the actual location of the starting credits, advertisement break and closing credits, respectively. The vertical dashed lines indicate the change points determined by Rupture. They appear to be very close to the genuine change points at the edges of the red blocks.

\begin{figure}[H]
\centering
\includegraphics[width=1\textwidth]{../plots/episode-pen100.png}
\caption{\texttt{Pelt} output for Figure \ref{fig:intro_ads_outro} data}
\label{fig:ruptures_change_detection}
\end{figure}

\subsection{Results} \label{sec:results}

\section{Discussion} \label{sec:discussion}

\section{Conclusions} \label{sec:conclusions}
